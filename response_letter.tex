\documentclass[11pt, letterpaper]{article}
\usepackage[top=1in, bottom=1in, left=1in, right=1in]{geometry}

% Math, graphics, and bibliography
\usepackage{amsmath,amssymb,amsfonts,mathrsfs,mathtools}

% My preferred fonts
\usepackage[T1]{fontenc}
\usepackage{lmodern}
\usepackage[sc]{mathpazo}
\usepackage{textcomp}

\usepackage{color}
\usepackage{tcolorbox}
\tcbuselibrary{breakable}
\tcbuselibrary{skins}
\usepackage{units}

\usepackage[parfill]{parskip} % For block paragraphs
\definecolor{gray}{gray}{0.4}
\newenvironment{reviewer}{\itshape\color{gray}}{}

\tcbset{skin=enhanced}

\newenvironment{manuscript}[1]{\begin{center}\begin{tcolorbox}[colback=green!5!white,colframe=green!75!black,width=0.8\textwidth,title={#1},breakable,fonttitle=\bfseries]}{\end{tcolorbox}\end{center}}


\begin{document}
\today\\

{\itshape Francis J. Doyle III}\\
Dept.\ of Chemical Engineering\\
Univ.\ of California, Santa Barbara\\
Santa Barbara, CA 93106-5080\\

{\itshape Leah Edelstein-Keshet}\\
Editorial Board Member\\
Biophysical Journal\\

Thank you for your email on August 1st, 2014, inviting us to submit a revised version of our manuscript, ``Amplitude metrics for cellular circadian bioluminescence reporters'' (MS: 2014BIOPHYSJ304278) by Peter C. St. John, Stephanie R. Taylor, John H. Abel, and Francis J. Doyle~III.

We thank the reviewers for their detailed comments of the manuscript and suggestions for how to improve the work. We did our best to balance the suggestions of Reviewer \#1, who suggested trimming much of the theoretical foundations of the method, with the suggestions of Reviewer \#2, who recommended additional explanations and validations. Ultimately we believe that the changes have greatly strengthened our manuscript by demonstrating our method is generalizable to alternative models and by focusing the work on our novel contributions.

Please see our detailed responses to the reviewers below: 

\section*{Reviewer \#1}
\begin{reviewer}
In this manuscript the authors study the amplitude of oscillators with a focus on circadian rhythms.
Understanding the effect of perturbations of populations of oscillators is important to understand the mechanisms of circadian rhythms.
 
The authors start by describing phase response curves PRCs, ARCs and phase density equations.
They also describe practical approaches to computing these.
Using such a computed perturbed distribution, the authors show how to calculate an ``average oscillation'' of the population.

My main concern is that I cannot tell what is novel in the manuscript.
Since references are scant in the main body of the text, this is somewhat obscure.
For example, phase response curves have been discussed in numerous places, as have amplitude response curves.
The phase density methods and an associated Fokker-Planck equation (Eq.~31) also date back at least to Kuramoto and Stratonovich, with a lot of work following that.
For instance, the ``synchronization index'', $\rho$, is the order parameter in Kuramoto's treatment, and has been studied in hundreds of papers.
I guess this is well known, but some references should be given.
Stochastic oscillators at the population level, with a focus on PRCs, have been studied in a series of papers by Tanaka, Goldobin, Ermentrout and others - see also the standard, but somewhat dated book by Pikovsky, Rosenblum and Kurths.

Pages 4-14 of the manuscript therefore contain a lot of known results.
Some of the results may be new - for instance, I have not seen Eq.~(26) before - but the majority of the results presented here have been standard fare in the voluminous literature on stochastic oscillators.
Is the novelty here that these can be used to analyze data? If so, this is not well explained.
\end{reviewer}

We acknowledge that many of the mathematical formulas presented in the ``Results'' section have been used by many other studies: our intention was to introduce concepts and notations to readers who might be less familiar with the large body of previous literature.
The reviewer's point is a good one, however, and we have endeavored to draw as clear a line as possible between new concepts and derivations and those which come from the literature.
{\itshape\bfseries To this end, we have moved as much of the introductory derivations as possible to the ``Materials and Methods'' section}, introducing previous work on phase-diffusion models before proceeding with sensitivity-based methods for finding the ARC.
Additionally, we have included additional references for historical derivations of the results:
\begin{manuscript}{Pages 5-6}
  The synchrony of the population can be modeled using a probability density function $p(\theta, \tilde{t})$ that describes the probability of finding an oscillator at each phase {\bfseries (33)}.
The usefulness of probability density functions in describing the phase and amplitude responses of populations of circadian cells has been {\bfseries previously shown} (14).
As with all probability density functions:
\begin{equation}
  \int_0^{2\pi} p(\theta, \tilde{t}) \; d\theta = 1
  \tag{12}
\end{equation}
The shape of $p(\theta, \tilde{t})$ changes as the cells advance in time.
Stochastic effects cause the population to gradually desynchronize as slight cycle-to-cycle deviations are propagated throughout the population {\bfseries (34). For a infinite population of oscillators,} these effects are well-described by a {\bfseries Fokker-Plank} equation (35):
\end{manuscript}

\begin{manuscript}{Page 21}
\bfseries
33. Kuramoto, Y., 1984. Chemical Oscillations, Waves, and Turbulence. Springer-Verlag Berlin
Heidelberg.\\
34. Teramae, J.-n., and D.~Tanaka, 2004. {Robustness of the noise-induced phase synchronization in a general
  class of limit cycle oscillators.} \emph{Phys. Rev. Lett.} 93:204103.
\end{manuscript}

To address the point directly, we consider the following contributions of the manuscript as novel, and highlight these in the results section:

\begin{itemize}
  \item Our definition of transient amplitude change (at the single-cell level). This definition allows us to calculate and directly compare amplitude change using both finite-difference and ODE sensitivity methods.
  \item The subsequent derivation of the differential ARC, similar to derivations of the differential phase response curve.
  \item We derive a method to approximate a stochastic population of noninteracting oscillators using continuous methods, including accounting for deviations from limit cycle trajectories.
\end{itemize}

While we have left some introductory derivations in the Results section (notably, derivations of the PRC and ARC for phase-only models), these sections have been clearly labeled:

\begin{manuscript}{Page 11}
\bfseries
We next briefly describe how to calculate the PRC and ARC for a phase-density model.
These approaches are commonly used in understanding phase models (14, 33), and are presented here to match previous definitions for single cells.
\end{manuscript}


\begin{reviewer}
The interesting part of the manuscript are the application to experimental data.
However, this part of the manuscript is very brief.
It is also not clear to me what methods discussed in the first part of the manuscript were used here.
An ARC was computed, and the phase density is also found for the model.
However, details on the connection to the first part of the manuscript are sparse.
\end{reviewer}

We have followed the reviewer's suggestion and expanded the section of the manuscript that focuses on the application to experimental data.
For instance, we detail our method of selecting the parameter knockdown as well as the initial phase through the use of response curves:

\begin{manuscript}{Page 16}
{\bfseries
The next step in capturing the experimental data was finding a perturbation capable of desynchronizing the system.
To find such a perturbation, we calculated population-level ARCs at several different knockdown strengths for each parameter in the model.
We selected the} degradation rate of {\itshape Per} mRNA {\bfseries from the feasible parameters}, due to PER's known induction by CREB following photo-perturbation (46).
\end{manuscript}

We have also expanded Fig.~5 to include response curves for the first parameter pulse, demonstrating the strong dip in the population-level ARC:

\begin{manuscript}{Page 17}
\bfseries
Figure 5: Method allows direct comparison between model results and experimental bioluminescence profiles.
({\bfseries A}) Response curves for the parameter perturbation used to model the first light pulse in (B).
The initial phase of the system is chosen to coincide with the strong dip in the population-level ARC (left). \dots
\end{manuscript}


\begin{reviewer}
Ultimately, I am not sure what the point of the manuscript is. The authors state that 
 
``\dots we provide new tools and definitions to quantify amplitude change in a population of oscillators, and determine what factors dominate amplitude change under certain conditions.''
 
Then I disagree - most of the tools here are not new.
Density methods for uncoupled populations of oscillators have been around for many years - at least since Stratonovich.
What is novel is potentially the treatment of the amplitude response curve.
If this is indeed the novel contribution, then this needs to be better explained.
\end{reviewer}

We acknowledge that our placement of the introductory material for the phase density methods likely gave the impression that our major results were already well-known.
We believe our revised manuscript gives a better emphasis to the novel aspects of the study, which we highlight in the introduction:

\begin{manuscript}{Page 4}
In this study, {\bfseries we describe approaches} to quantify amplitude change in a population of {\bfseries non-interacting} oscillators.
{\bfseries By exploiting the independence of each oscillator, we derive computationally efficient methods to approximate the mean dynamics of full stochastic simulations.
Additionally, our method allows the calculation of ARCs at both the single-cell and population level, allowing the behavior of the system to be quickly profiled.}
\end{manuscript}

\begin{reviewer}
I suggest rewriting the manuscript with an emphasis on the application.
It is still a bit unclear how the proposed method can be applied to real data, and what one would get out of it.
Therefore a focus on these ideas, rather than the review of results about PRCs, ARCs and phase densities, would significantly improve the manuscript.
\end{reviewer}
 
We have modified extensive portions of the manuscript in order to make the novel contributions of the manuscript more clear, and added additional explanations to the application of the experimental data:

\begin{manuscript}{Page 16}
From the experimental system, it appears the light pulse temporarily reduces PER2-luciferase amplitudes immediately following perturbation, separate from the overall change in population synchrony.
{\bfseries This result is most obvious after the second light pulse, where rhythms require some time before their maximum amplitude is reached.
Such a delay} is indicative of perturbations to the {\bfseries limit cycle system, suggesting} a contribution from single-cells in determining the population amplitude.
{\bfseries Light-induced amplitude suppression in fibroblasts is therefore likely mediated at the single-cell level at short time-scales, and at the population level for longer time-scales.
To improve the fit of the model to the data, the response curves could be fit to match experimentally predicted values.
Infinitesimal PRCs and ARCs calculated at the single-cell level are numerically efficient to evaluate, and could be more readily incorporated into a parameter estimation algorithm than explicit stochastic simulations of a population of oscillators.}
\end{manuscript}

Additionally, {\itshape\bfseries many details on the derivations on PRCs have been removed.}

\begin{reviewer}
Minor comments: 
- Equation 11.
Could $\Theta(x_0)$ be defined as $\theta$ such that $lim_{t\rightarrow \infty} | x(t)-x^\gamma(\theta+t)|=0$?
\end{reviewer}

Yes, although for length constraints (and to remove excess review of results on PRCs) {\itshape\bfseries this equation has been removed.}

\begin{reviewer}
- Equation 16: Could $\mu$ just be defined as $\int_0^{2\pi} y(t)dt/(2\pi)$ ?
\end{reviewer}

Yes, and we have changed this definition accordingly:

\begin{manuscript}{Page 7}
\begin{equation}
  \mu \coloneqq \int_0^{2\pi} \frac{x^\gamma(\theta)}{2\pi} \; d\theta
  \tag{21}
\end{equation}
\end{manuscript}

Our original definition was intended to demonstrate why $x(t)$ and $y(t)$ would have identical means, but we have simplified the definition to the one suggested.

\begin{reviewer}
- Equation 27 is incomplete. 
\end{reviewer}
 
We have fixed this regrettable error, and thank the reviewer for their careful reading:

\begin{manuscript}{Page 10}
\begin{equation}
  \frac{d}{d\tilde{t}}\frac{dA}{dp} \coloneqq \lim_{\tilde{d},\; \Delta p \to 0} \frac{\Delta A}{\tilde{d}\, \Delta p}
  \tag{31}
\end{equation}
\end{manuscript}

\begin{reviewer}
- In Figure 5A, the initial condition for the experiments is 1. Why did the authors use the initial condition $0.5$ in the simulations?
\end{reviewer}

The initial conditions of the experiments likely arises from a normalization of the bioluminescence data, and absolute values were not available to directly compare amplitudes.
Therefore the relative amplitudes of both systems had to be compared, with the amplitude estimated from the control trajectory.
The phase offset of the experimental and model systems was also not fit exactly - we added an explanation of the rationale behind this decision to the manuscript:

\begin{manuscript}{Page 16}
\bfseries
The single-cell and population-level response curves for this parameter choice are shown in Fig.~5A, which demonstrates the strong desynchronization that occurs at $\theta\approx\nicefrac{\pi}{4}$.
Since the reporter used in the experimental system likely has a phase lag from the corresponding mRNA, we did not try to match the initial phase of the simulation to experiment.
Instead, the initial phase of simulation was chosen such that the first pulse occurred when the system was at the phase that corresponded with the minimum of the population-level ARC.
\end{manuscript}

\section*{Reviewer \#2}

\begin{reviewer}
The manuscript titled ``Amplitude metrics for cellular circadian bioluminescence reporters'' investigates a very interesting problem: amplitude perturbations via changes in single-cell levels and/or population synchrony.
Furthermore, the manuscript suggests that amplitude changes mediated by single-cell level or population-level changes can be distinguished based on the analysis in the manuscript.
I totally agree that this is a very important problem in circadian clocks as well as many other biological oscillators.
However, unlike the statement in the abstract of the manuscript ``In particular, our analysis reveals that amplitude changes mediated by individual cells or population synchrony can be distinguished in tissue-level bioluminescence data without the need for single-cell measurements'', I am not convinced how their analysis provides such a tool.
Furthermore, I found some mistakes in their derivation of a major formula, which reduces the reliability of the manuscript.
Finally, I am also concerned about major simulation results because the Gillespie algorithm with non-elementary propensity functions is used, which can cause a large error.
In summary, I strongly agree that this manuscript considers an interesting and important problem, but due to the above reasons, I am not convinced with their results.
To resolve these issues I have the following suggestions: 
 
Major concerns 
 
1) During the derivation of the major formula (Eq 18- 28), I found some errors. Please, fix them.
\end{reviewer}

We thank the reviewer for their careful review of the mathematical derivations. The errors are largely typographical:

\begin{reviewer}
a. (21) and (22) are not equal. 
\end{reviewer}

This mistake has been corrected.
Eq.~22 (now 27) is obtained by factoring the difference of squares from the previous equation, but we mistakenly kept a squared term from the previous step:

\begin{manuscript}{Page 10}
  \begin{equation}
  = \frac{1}{\delta x(0)} \left[\left(\delta x(\tilde{t}) -
    \tilde{f}\left(x^\gamma(t_\theta)\right)\delta\theta\right) \left(\delta
    x(\tilde{t}) + \tilde{f}\left(x^\gamma(t_\theta)\right)\delta\theta +
    2(x^\gamma(t_\theta) - \mu)\right) \right]
    \tag{27}
  \end{equation}
\end{manuscript}

\begin{reviewer}
b. (22) and (23) are not equal. (23) is an approximation of (22) when delta theta approaches to zero. 
\end{reviewer}

This is a fair point, as we dropped the infinitesimal elements before taking the limit.
The equality is valid in the differential limit as the phase shift for an infinitesimal perturbation will be identically 0, and therefore this error in notation does not affect the validity of the results.
However, including the delta theta terms explicitly made the equation too long for a single line.
Since the equation (with the delta theta terms included) was largely a repetition of the previous equation, {\bfseries\itshape we have removed it.}

\begin{reviewer}
c. (23) and (24) are not equal.
\end{reviewer}

This typo was also fixed by removing the squared term in Eq.~24 (now 29), which was unfortunately propagated from the typo in Eq.~22.

\begin{manuscript}{page 10}
  \begin{equation}
  = 2\left(S(\tilde{t}) - \tilde{f}(x^\gamma(t_\theta))\frac{d\theta}{dx}\right)\left(x^\gamma(t_\theta) - \mu\right)
    \tag{29}
  \end{equation}
\end{manuscript}

\begin{reviewer}
d. (27) is not complete.
\end{reviewer}

{\itshape\bfseries This error has been fixed}, as noted in our response to Reviewer \#1.

\begin{reviewer}
e. Please, provide an intuitive explanation for formulas (26) and (28) for the reader. 
\end{reviewer}

The following two explanatory remarks have been added following Eq.~26 and 28 (now 30 and 32):

\begin{manuscript}{Page 10}
  {\bfseries Here, the first term of the integrand $(S - \tilde{f}\dot{\theta})$ tracks the distance from the perturbed trajectory to the limit cycle, which decays to zero as $t \to \infty$.
The second term $(x^\gamma - \mu)$ weights this distance by whether or not the deviation occurs above or below the oscillatory mean, yielding negative amplitude changes when the trajectory is perturbed closer to the mean.}
Just as with the parameter-impulse PRC, the infinitesimal parameter-impulse ARC {\bfseries is defined as}:
\begin{equation}
  \frac{d}{d\tilde{t}}\frac{dA}{dp} \coloneqq \lim_{\tilde{d},\; \Delta p \to 0}
  \frac{\Delta A}{\tilde{d}\, \Delta p}
  \tag{31}
\end{equation}
{\bfseries and} may be calculated from the state-impulse version with the following relationship:
\begin{equation}
  \frac{d}{d\tilde{t}}\frac{dA}{dp} = \frac{dA}{dx}\frac{d\tilde{f}}{dp}
  \tag{32}
\end{equation}
{\bfseries As with Eq.~11, this equivalency reflects the fact that for a pulse of infinitely short duration, a parameter change is equivalent to changing the state of the system along the direction specified by the Jacobian.}
\end{manuscript}

\begin{reviewer}
2) Whereas the manuscript describes that they explore ``weakly coupled oscillators'', I found that what the manuscript is really considering are ``independent or non-interacting oscillators''.
Weakly coupled oscillators have been used in a different context (e.g.\ Kuramoto oscillators).
Please, correct this.
\end{reviewer}
 
{\bfseries\itshape The terminology has been changed throughout the manuscript.}
We thank the reviewer for this correction.

\begin{reviewer}
3) One of the most interesting results of this manuscript (Fig4) is showing that ARCs, but not PRCs are very different at the single-cell level as compared to the population level.
I can see that a single simulation (Fig.~4A) supports this, however, I cannot see that the analysis in the manuscript is sufficient to support this statement.
That is, please explain how the analysis of this manuscript (perhaps the formula for the ARC) leads to the conclusion that appears in Fig.~4A.
Furthermore, I wonder if the result in Fig.~4A is a specific example of a special case or a general property.
In particular, a recent study (DeWoskin et al (2014)) shows that PRCs can also be very different in single-cell and population levels.
To test the generality of the results (Fig.~4A), please test this argument, at least with more simple models such as Model 1 in the manuscript, a Goodwin oscillator based on a Hill function, or the Kim-Forger model based on protein sequestration.
\end{reviewer}
 
Our point in discussing the phase response curves of Fig.~4A is only that phase maintains a consistent definition between the population and individual cell level, while amplitude must be defined differently in each case.
We have added the following text to the manuscript to make this point more explicit:

\begin{manuscript}{Page 14}
  {\bfseries In this case,} the population-level phase change is a slightly smoothed version of the single-cell {\bfseries PRC}, since the population has an averaging effect on incoming perturbations with each cell receiving the input at a slightly different internal phase.
{\bfseries This smoothing follows from the consistent definition of phase between the single-cell and population level.
In the ARC plot, however, the shape of the single-cell and population-based ARC are different, since they describe different types of changes (finite vs.\ sustained) in the output trajectories.}
\end{manuscript}

The PRC result in Fig.~4A is not a special case, but other types of phase-resetting behavior in a population can occur.
For instance, an almost desynchronized population would undergo type 0 (strong) phase resetting, even if the PRC of each cell was type 1 (weak), since any change to the phase-density profile would create a new mean phase.
Ultimately, we feel that the literature on the collective phase-response of a population of oscillators (coupled or non-coupled, simulated either explicitly or with phase-density methods) is relatively mature, and that a providing a detailed analysis of the phase responses we obtain is beyond the scope of our major contribution (outlined in response to Reviewer 1).
 
\begin{reviewer}
4) The author suggests that ``amplitude change induced at the population level will follow a consistent decaying sinusoid, while a change induced at the single-cell level will deviate from this exponential decay''.
Is this a general statement or the conclusion of a specific simulation result (Fig.~4B and C)? If this is not a general statement, then I doubt whether this result can be used to distinguish two sources of amplitude change without explicit recording of single cell amplitudes.
If this is a general statement, please provide more details and intuitive explanations, as well as further simulation results with more types of models, or analysis.
\end{reviewer}
 
{\bfseries\itshape This is a general statement} regarding the dynamics of populations and limit-cycle systems. Ultimately, non-interacting oscillators have no driving force to re-entrain, so changes to synchrony remain long after limit-cycle oscillations are restored.

In addition to adding simulation results with additional models (see below), we have changed the quoted text to be more explicit in this explanation:

\begin{manuscript}{Page 14}
  \bfseries
These examples demonstrate the qualitative differences between amplitude change mediated at the single-cell and population level.
Their characteristic features allow us to distinguish between these two sources of amplitude change without explicitly recording single-cell amplitudes: amplitude change induced at the population-level will be sustained, but may be masked in the short term by changes to single-cell level amplitudes.
Similarly, a change induced at the single-cell level will be evident only at short time-scales.
These results underscore the importance in considering both single-cell and population-mediated amplitude change in predicting the effects of daily stimuli on clock amplitudes.
\end{manuscript}

\begin{reviewer}
5) To derive the major formula in Eq.~(50), various approximations and assumptions were made.
This formula is tested with a single simulation.
To confirm the accuracy of Eq (50), please, test this in other models as suggested above.
\end{reviewer}

(Answered below)

\begin{reviewer}
6) Stochastic simulations of the model with non-elementary reactions (e.g.
Hill-functions or Michaelis-Menten terms) are done with the Gillespie algorithm.
However, recent studies have shown the inaccuracy of this approach (Thomas et al.
BMC Syst Biol (2012) and Agarwal et al.\ J Chem Phy (2012)).
Please, discuss whether this inaccuracy in the stochastic simulations affects the simulation results in Fig.~5.
If possible, please try a model based on mass-action kinetics or a model based on total QSSA since Yao et al.\ Biophy J (2008) shows that the model based on total QSSA can provide accurate simulations with the Gillespie algorithm.
\end{reviewer}
 
Points 5 and 6 are linked concerns regarding the simulation and approximation of stochastic trajectories, so we have addressed them together.
The noise generated using the Gillespie algorithm for non-elementary propensity functions can indeed be different than the noise that would occur for the full system when expanded into elementary reactions.
That being said, the level of noise present in Fig.~5 was chosen by varying the volume parameter of the stochastic simulation so that desynchronization rates matched experimentally observed data.
We therefore do not believe that incorrect noise estimates from the Gillespie algorithm affects our approximations of the stochastic population.
{\itshape\bfseries Regardless, we tested the method with two additional models.}
First, the Oregonator, is a simple mass-action oscillatory reaction system considered in Gillespie's 1977 paper.
We used a perturbation to the conversion rate of two of the intermediate species at 6 different phases to test the accuracy of the perturbed approximation, $x_hat$.
Second, we tested the method with a stochastic simulation of a more detailed circadian model.
For consistency, we used the same model and perturbation as was used in Fig.~5 (for the first light pulse).
The results of these simulations are included as Fig.~S3, which (similar to Fig.~S2) demonstrates good agreement between the stochastic models and continuous approximations.

\begin{manuscript}{Supplemental Info, Page 6}
\bfseries
Figure S3: Validation of continuous methods with alternate models.
(\bfseries A) The Oregonator model (Model 3) was tested as an example of a mass-action limit cycle oscillator.
A population of 2000 oscillators was perturbed at size different mean phases, $\mu$, by increasing the $\mathit{c2}$ parameter by $50\%$ for $d=\pi$ (highlighted region).
Only the average $\mathit{Y2}$ variable is plotted.
({\bfseries B}) The model described in (2) (Model 2) was similarly tested at a variety of mean phases by reducing the $\mathit{vdp}$ parameter by $28.5\%$ and plotting the resulting $\mathit{c2}$ state variable (as in Fig.~5).
Parameters for the continuous approximations were found by estimating the phase, initial standard deviation, period, and phase diffusivity of the control population.
Good agreement is seen between the continuous and stochastic simulations, indicating the proposed method is suitable for predicting the population-level responses for a variety of model types.
\end{manuscript}
 
\begin{reviewer}
Minor concerns: 

1) In most case, ``hat'' mark is used to represent various perturbed variables.
However, ``hat'' mark is also used to represent the rescaled time variable.
This can lead to confusion in understanding the equations.
Please, use different notation for scaled time.
\end{reviewer}

This is a good suggestion, and {\itshape\bfseries we have changed the notation for scaled time to $\tilde{t}$ throughout the manuscript.}
Additionally, we noticed that our notation for pulse duration conflicted with our definition of phase diffusivity, so pulse duration has been changed to $\tilde{d}$.

\begin{reviewer}
2) Please provide a little more detail for the derivation of (14).
\end{reviewer}

The derivation for Eq.~14 (now 11) is analogous to Eq.~28 (now 32), a lengthy process that was detailed in the appendix of Taylor et al, 2008 (PMID 19593456).
For space constraints, we have added only brief remarks on their derivation:

\begin{manuscript}{Page 5}
  Methods for efficiently calculating these quantities using ODE sensitivity analysis have been developed (24), with the important result that {\bfseries the parameter- and state-impulse PRCs can be related by the Jacobian matrix:}
\begin{equation}
  \frac{d}{d\tilde{t}}\frac{d\theta}{dp} = \frac{d\theta}{dx}\frac{d\tilde{f}}{dp} 
  \tag{11}
\end{equation}
{\bfseries This result follows from the fact} that in the limit {\bfseries of an infinitely short and small parameter pulse, $\Delta x(0) \to \nicefrac{d\tilde{f}}{dp}\,\tilde{d}\, \Delta p$.
This result follows from the fact that in the limit of an infinitely short and small parameter pulse, $\Delta x(0) \to \nicefrac{d\tilde{f}}{dp}\,\tilde{d}\, \Delta p$.}
\end{manuscript}

\begin{reviewer}
3) In Fig.~2A, a discontinuous PRC appears as continuous curve.
Please, do not link the discontinuous points, and redraw the PRCs.
\end{reviewer}

{\bfseries\itshape This plot has been corrected.}

\begin{reviewer}
4) The PRC and ARC in Fig 4A are in response to which perturbation?
Please provide details on how Fig 4A is simulated.
\end{reviewer}

Additional details on the simulation of Fig. 4 have been added to both the main text and the figure legend:

\begin{manuscript}{Page 14}
{\bfseries We next demonstrate the usefulness of single-cell and population level ARCs in predicting the mean response of a population.
  Using a detailed single-cell model (Model 2), we calculate the response of the system, specifically the average CRY2 protein expression level, to a temporary increase in {\itshape Per} transcription rate.}
In Fig.~4A, we plot {\bfseries the resulting} single-cell and population-level response curves.
\end{manuscript}

\begin{manuscript}{Page 15}
  Figure 4: Response curves {\bfseries describe different types of amplitude change
A population of cells is simulated using a detailed model of circadian rhythms (Model 2).
Each cell in the population is subjected to a $20\%$ increase in the $\mathit{vtp}$ parameter ({\itshape Per} transcription rate) for $d=\nicefrac{\pi}{2}$, inducing both a phase and amplitude change.}
\end{manuscript}

\begin{reviewer}
5) The definition of Eq (48) is not clear.
Please provide a more detailed explanation.
\end{reviewer}

Eq~ 48 (now 44) tracks the distance between the perturbed trajectory and the limit cycle for each phase.
The weighted-average of this distance is then used to find the population-level expression following perturbation.
To make this point more clear, we have added additional explanatory text to the manuscript:

\begin{manuscript}{Pages 12-13}
The second contribution to the perturbed population trajectory comes from deviations from limit cycle oscillations in each cell.
{\bfseries We calculate the population-level effect of these deviations by averaging over the deviations that occur at each phase.}
We define the deviation trajectory $\delta x(\theta, \tilde{t})$ for each phase as the distance between the perturbed trajectory and the phase-adjusted reference:
\begin{align}
  \delta x(\theta_0, \tilde{t}) &\coloneqq x(\tilde{t}) - x^\gamma(\tilde{t} + \theta_0 + \Delta \theta)\tag{48} \\
  \therefore\; \lim_{\tilde{t} \to \infty} \delta x(\theta, \tilde{t}) &= 0 \tag{49}
\end{align}
{\bfseries Since the perturbed trajectory ultimately converges with the phase-adjusted reference, deviations will converge to zero.
Since the phase change, $\Delta\theta$, associated with a perturbation at each phase is likely not known prior to calculating the perturbed trajectory, it is difficult to tabulate deviation trajectories associated with each final phase, $\theta_0 + \Delta\theta$.
It is therefore more straightforward to find the average effect of single-cell perturbations at the population level by weighting the deviations by the phase density function prior to perturbation.}
\end{manuscript}

\begin{reviewer}
6) In a couple of places, the manuscript discusses pharmacological manipulation of circadian clocks.
Could you discuss this in more detail since the manuscript analyzes data coming from only light-induced perturbations (Fig.~5)? 
For instance, the manipulation of circadian rhythms with the CK1d/e inhibitor (PF-670462) has been studied widely.
Can you discuss how your method can be applied to the data in these studies?
\end{reviewer}

Indeed, while we would have preferred to analyze data coming from pharmacological perturbations, this type of data is difficult to obtain.
We have added an additional paragraph to the conclusion to address this limitation, and point out potential future applications of the method:

\begin{manuscript}{Page 18}
\bfseries
While our examples have focused on light-mediated perturbations, these approaches could also be applied to pharmacological perturbations.
Some difficulty arises in obtaining such data in cultured cells, however, as entraining perturbations would require that pharmacological agents be introduced for only a finite duration.
As medium or temperature changes are often enough to resynchronize cultured cells, a transient application is difficult to achieve experimentally.
The search for clock-enhancing molecules has therefore tended to focus on constant drug concentrations: for instance, dose-dependent period or amplitude change following inhibition of CK1$\delta$ or similar targets (48).
For {\itshape in vivo} systems, however, pharmacokinetics dictates a finite duration of action for both naturally secreted hormones and pharmacological therapies.
More effective treatments might therefore be designed by explicitly accounting for such a transient response, perturbing peripheral clocks at the right phase to induce resynchronization and an increase in single-cell level amplitudes.
\end{manuscript}

\section*{Reviewer \#3}

\begin{reviewer}
Many circadian clock studies use reporters and analyze ensembles of cells.
Frequently phase shifts are analyzed using Phase Response Curves.
The authors complement this approach by introducing Amplitude Response Curves.
Moreover, they discuss effects of ensemble averaging.
A clear mathematical description is provided which is useful to interpret experimental data.
Some related studied should discussed to embed the manuscript and to highlight the novelty.

\end{reviewer}
 
We thank the reviewer for their positive impression of the manuscript, as well as for their helpful suggestions of related studies to discuss. We have incorporated many of the suggested works, which we feel helps to place the manuscript within the larger field of circadian modeling.

\begin{reviewer}
Detailed comments: 
 
Felix Naef studied stochastic models to discuss fibroblast rhythms.
He found as well purely exponential decay in case of damped oscillators and deviations for limit cycles.
The current study should be compared to these models.
\end{reviewer}
 
In the relevant studies from Felix Naef, the authors use a phase model in which Gaussian white noise causes the frequencies to drift in time.
The decay rate is estimated from stochastic simulations of detailed limit cycle models, and used to capture unperturbed bioluminescence data.
We have incorporated two relevant references:

\begin{manuscript}{Page 21}
{\bfseries
32. Rougemont, J., and F. Naef, 2006. Collective synchronization in populations
of globally coupled phase oscillators with drifting frequencies. {\itshape Phys.\ Rev.\ E} 73:011104.\\}
\dots\\
{\bfseries
  36. Rougemont, J., and F. Naef, 2007. Dynamical signatures of cellular fluctuations and oscillator stability in peripheral circadian clocks. {\itshape Mol.\ Syst.\ Biol.\ } 3:93. }
\end{manuscript}

and added the references to the following sections of the manuscript:

\begin{manuscript}{Page 5}
\bfseries
Large populations of oscillators are typically described using phase-only models (32), in which the state of each oscillator is represented only by its phase, $\theta$.
\end{manuscript}
\begin{manuscript}{Page 6}
The phase diffusivity parameter $d$ (in units of inverse radians) describes the speed with which the population desynchronizes and can be fit to experimental data {\bfseries (36)}.
\end{manuscript}
\begin{manuscript}{Page 11}
\bfseries
Previous work has used limit cycle models to estimate population-level parameters, such as desynchronization rate, for phase-only models (36).
\end{manuscript}
\begin{manuscript}{Page 12}
  Phase-diffusion {\bfseries models can explain} why the gradual damping from experimental population-level data closely resembles an exponentially damped sinusoid, {\bfseries a result that has been shown experimentally (9) and computationally (36).}
\end{manuscript}

\begin{reviewer}
Why red and blue are changed to orange in Figure 2D? 
\end{reviewer}
 
In Fig.~2D, we plot the time-dependent state variables of the two-state model (Model 1).
These variables had not been explicitly plotted in this representation in another figure, so we wished to distinguish them from state-space representations or response curve variables.

\begin{reviewer}
David Rand published a generic theory of response functions which should include ARC?
A comparison is suggested. 
\end{reviewer}

Indeed, David Rand's generic theory of response functions would likely include our amplitude response curve as an example of a differentially changed quantity following a perturbation of finite duration.
Our contribution lies in the definition and derivation of the relevant formulas for amplitude change.
We have included the relevant citation from David Rand, along with a discussion in the manuscript text:

\begin{manuscript}{Page 3}
Mathematical models have long been used to understand the results of circadian experiments (15, 16), {\bfseries aided by} definitions and computational techniques {\bfseries designed to match modeling predictions to experimental data.
One such definition is the response function, a general technique that maps a change in an output variable to a temporary change in parameters (17).}
For instance, the phase response curve (PRC) has been used to characterize the entrainment behavior of both experimental and mathematical systems (18-20), and in analyzing the synchrony of populations of oscillators (21).
\end{manuscript}

\begin{reviewer}
Joke Meijer (PLoS One) discussed the comparison of single cell amplitudes and ensembles of neuron regarding resetting dynamics as well. 
\end{reviewer}
 
Thank you for drawing our attention to this work, as it lends some experimental support to the bimodal distributions we observe following a light pulse.
We have incorporated the citation into the relevant section of the manuscript:

\begin{manuscript}{Page 14}
Changes to the synchronization of the population from each perturbation are readily apparent, {\bfseries with both perturbations inducing a bimodal distribution in the phase density.
While experimental data on individual cell phases was unavailable for this data set, bimodal distributions in SCN neuron firing following a phase shift have previously been seen experimentally (47).
As higher-frequency features, these bimodal distributions} dissipate as the phases diffuse.
\end{manuscript}

\begin{reviewer}
The critical role of amplitudes for the response to perturbations has been discussed several times in the literature and a discussion of these previous studies can support the relevance of the current manuscript.
Examples: Pittendrigh JBR 1991, RK Barrett J.~Neuroscience 1995, MH Vitaterna PNAS 2006, S Brown PNAS 2008, A Granada PLoS One 2013, A.~Erzberger Proc Royal Soc 2013.

\end{reviewer}

The importance of amplitude in entrainment is an important consideration and has certainly been addressed by many studies in the past.
We have incorporated several of these references into the motivation for the manuscript, as designing therapies to boost circadian amplitudes will surely have an effect on the entrainment of the oscillators as well:

\begin{manuscript}{Page 2}
  {\bfseries The amplitude of circadian transcription is a relevant factor, and has been shown to play a critical role in phase resetting and entrainment (3, 4).}
Recent studies have further highlighted the importance of high peripheral clock amplitudes in maintaining metabolic health.
\end{manuscript}

\begin{reviewer}
Ute Abraham et al. (Molecular Systems Biology) compare also coupled cells (SCN) with less coupled cells (lung) regarding their entrainment properties and discuss amplitude and relaxation effects.
\end{reviewer}

We have incorporated this reference into two places in the manuscript. First, in the introduction as reference (4) noted above, and also in the conclusion where we discuss the relative entrainment speed of coupled vs. uncoupled cells:

\begin{manuscript}{Page 18}
It is therefore interesting that liver cells have not developed a stronger mechanism of intercellular coupling, such as in the SCN, to maintain robust amplitudes in the absence
of external cues.
Perhaps the ability to quickly re-entrain to a phase shift, which is often slowed by coupling {\bfseries (4)}, has historically been more advantageous than protection against highly variable food intake.
\end{manuscript}


\end{document}
