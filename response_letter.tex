\documentclass[11pt, letterpaper]{article}
\usepackage[top=1in, bottom=1in, left=1in, right=1in]{geometry}

% Math, graphics, and bibliography
\usepackage{amsmath,amssymb,amsfonts,mathrsfs,mathtools}

% My preferred fonts
\usepackage[T1]{fontenc}
\usepackage{lmodern}
\usepackage[sc]{mathpazo}
\usepackage{textcomp}

\usepackage{color}
\usepackage{tcolorbox}
\tcbuselibrary{breakable}
\tcbuselibrary{skins}
\usepackage{units}

\usepackage[parfill]{parskip} % For block paragraphs
\definecolor{gray}{gray}{0.4}
\newenvironment{reviewer}{\itshape\color{gray}}{}

\tcbset{skin=enhanced}

\newenvironment{manuscript}[1]{\begin{center}\begin{tcolorbox}[colback=green!5!white,colframe=green!75!black,width=0.8\textwidth,title={#1},breakable,fonttitle=\bfseries]}{\end{tcolorbox}\end{center}}


\begin{document}
\today\\

{\itshape Francis J. Doyle III}\\
Dept.\ of Chemical Engineering\\
Univ.\ of California, Santa Barbara\\
Santa Barbara, CA 93106-5080\\

{\itshape Leah Edelstein-Keshet}\\
Editorial Board Member\\
Biophysical Journal\\

Thank you for your email on September 24th, 2014, inviting us to submit a revised version of our manuscript, ``Amplitude metrics for cellular circadian bioluminescence reporters'' (MS: 2014BIOPHYSJ304278R) by Peter C. St. John, Stephanie R. Taylor, John H. Abel, and Francis J. Doyle~III.

We thank the reviewers for their second round of comments of the manuscript, in particular Reviewer \#2 for their continued attention to detail. We believe addressing these remaining concerns and suggestions have improved the polish of the manuscript.

Please see our detailed responses to the reviewers below: 

\section*{Reviewer \#1}
\begin{reviewer}
The authors have largely responded you my concerns. 
 
Typos and suggestions 
 
- p. 2: "fail to capture the collective dynamics of a population oscillators" 
\end{reviewer}

This typo has been addressed, the sentence now reads:

\begin{manuscript}{Page 2}
  Ordinary differential equation (ODE) models of gene regulation are capable of describing the amplitude and phase-resetting behavior of single cells, but fail to capture the collective dynamics of a population {\bfseries of} oscillators.
\end{manuscript}
 
\begin{reviewer}
- mu is defined a few lines after it is first used in Eq. (20). Perhaps define it first? 
\end{reviewer}

This change has also been made.

\begin{manuscript}{Page 7}
To define an amplitude change metric for such a case, we compare a perturbed trajectory $x(t)$ to a phase-shifted limit cycle reference $y(t)$, for which $x(t) \to y(t)$ for sufficiently long times.
Since $x(t)$ approaches the reference as $t \to \infty$, {\bfseries the means of both trajectories are equal and can be calculated by
\begin{equation}
  \mu \coloneqq \int_0^{2\pi} \frac{x^\gamma(\theta)}{2\pi} \; d\theta.
  \tag{20}
\end{equation}
The amplitude change metric is defined as}
\begin{equation}
  \begin{aligned}
    \Delta A (x(t), y(t)) &\coloneqq \int_0^\infty (x(t) - \mu)^2 - (y(t) - \mu)^2 \; dt\\
    &= \int_0^\infty h(t) \; dt.
  \end{aligned}
  \tag{21}
\end{equation}
\end{manuscript}


\end{document}
