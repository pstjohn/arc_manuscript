\documentclass[11pt, letterpaper]{article}
\usepackage[top=1in, bottom=1in, left=1in, right=1in]{geometry}


% Math, graphics, and bibliography
\usepackage{amsmath,amssymb,amsfonts,mathrsfs}
\usepackage{graphicx}
\usepackage[round,numbers,sort&compress]{natbib}
\renewcommand{\bibnumfmt}[1]{#1.}
% \usepackage[square,sort,comma,numbers]{natbib} % my preferred formatting

% My preferred fonts
\usepackage[T1]{fontenc}
\usepackage{lmodern}
\usepackage[sc]{mathpazo}

% For fancier fractions and figure captions
\usepackage[labelfont={bf}, margin=1cm]{caption}
\usepackage{units}

% For model float
\usepackage{float}
\newfloat{model}{thp}{lop}
\floatname{model}{Model}
\floatstyle{plaintop}
\restylefloat{model}
\usepackage{supertabular}
\usepackage{booktabs}

% Puts the affiliations on the cover page
\usepackage{authblk}

\begin{document}
\title{Supporting Material for:\\ Amplitude metrics for cellular \\ circadian bioluminescence reporters}
\author[1]{Peter C. St. John}
\author[2]{Stephanie R. Taylor}
\author[1]{John H. Abel}
\author[1,*]{Francis J. Doyle III}
\affil[1]{Department of Chemical Engineering, University of California Santa
Barbara, Santa Barbara, California 93106-5080}
\affil[2]{Department of Computer Science, Colby College, Waterville, Maine
04901}
\affil[*]{Email: \texttt{doyle@engineering.ucsb.edu}}
\date{}
\maketitle

\begin{model}
\caption{
Adapted from Nov\'{a}k \& Tyson, 2008 \cite{Novak2008}. Model used for
ARC demonstrations in Figs. 1-3, S1 ($P=4$) and Fig. S2, Movies S1 \& S2
($P=2$).
It should be noted that the use of the stochastic simulation algorithm for reactions with non-elementary propensities may inaccurately represent the noise of the full system.
In this case, we simply fit the noise characteristics of the approximated model (using the volume parameter) to yield physiologically realistic desynchronization rates. The volume of the stochastic simulations was chosen as $\Omega = 250$.
}
  \centering
\begin{align*}
  \frac{dX}{dt} &= \frac{1}{1 + Y} - X \\
  \frac{dY}{dt} &= k_t \; X - k_d \; Y - \frac{Y}{\alpha_0 + \alpha_1 \; Y +
  \alpha_2 Y^2}
\end{align*}
  
\tablehead{\toprule Parameter & Value\\\midrule}
\begin{supertabular}{rl}
$k_t$ & 20 \\
$k_d$ & 1  \\
$P$   & 4 (or 2)  \\
\bottomrule\end{supertabular}\hspace{3ex}
\begin{supertabular}{rl}
$\alpha_0$ & 0.005 \\
$\alpha_1$ & 0.05  \\
$\alpha_2$ & 0.1   \\
\bottomrule\end{supertabular}
\end{model}

\begin{model}
\caption{Hirota {\itshape et al.} 2012 \cite{Hirota2012}. A more detailed model
of circadian rhythms used for simulations in Figs. 4-5.}
  \centering
\begin{align*}
\frac{d\,\mathit{p}}{dt} &= \frac{- \mathit{p} \cdot \mathit{vdp}}{\mathit{kdp} + \mathit{p}} + \frac{\mathit{vtp}}{\mathit{knp} + \left(\mathit{C1n} + \mathit{C2n}\right)^{3}}\\
\frac{d\,\mathit{c1}}{dt} &= \frac{- \mathit{c1} \cdot \mathit{vdc1}}{\mathit{c1} + \mathit{kdc1}} + \frac{\mathit{vtc1}}{\mathit{knc1} + \left(\mathit{C1n} + \mathit{C2n}\right)^{3}}\\
\frac{d\,\mathit{c2}}{dt} &= \frac{- \mathit{c2} \cdot \mathit{vdc2}}{\mathit{c2} + \mathit{kdc1}} + \frac{\mathit{vtc2}}{\mathit{knc1} + \left(\mathit{C1n} + \mathit{C2n}\right)^{3}}\\
\frac{d\,\mathit{P}}{dt} &= - \mathit{C1} \cdot \mathit{P} \cdot \mathit{vaC1P} + \mathit{C1n} \cdot \mathit{vdC1P} - \mathit{C2} \cdot \mathit{P} \cdot \mathit{vaC1P} + \mathit{C2n} \cdot \mathit{vdC1P} - \frac{\mathit{P} \cdot \mathit{vdP}}{\mathit{P} + \mathit{kdP}} + \mathit{ktxnp} \cdot \mathit{p}\\
\frac{d\,\mathit{C1}}{dt} &= - \mathit{C1} \cdot \mathit{P} \cdot \mathit{vaC1P} - \frac{\mathit{C1} \cdot \mathit{vdC1}}{\mathit{C1} + \mathit{kdC1}} + \mathit{C1n} \cdot \mathit{vdC1P} + \mathit{c1}\\
\frac{d\,\mathit{C2}}{dt} &= - \mathit{C2} \cdot \mathit{P} \cdot \mathit{vaC1P} - \frac{\mathit{C2} \cdot \mathit{vdC2}}{\mathit{C2} + \mathit{kdC1}} + \mathit{C2n} \cdot \mathit{vdC1P} + \mathit{c2}\\
\frac{d\,\mathit{C1n}}{dt} &= \mathit{C1} \cdot \mathit{P} \cdot \mathit{vaC1P} - \mathit{C1n} \cdot \mathit{vdC1P} - \frac{\mathit{C1n} \cdot \mathit{vdCn}}{\mathit{C1n} + \mathit{C2n} + \mathit{kdCn}}\\
\frac{d\,\mathit{C2n}}{dt} &= \mathit{C2} \cdot \mathit{P} \cdot \mathit{vaC1P} - \frac{\mathit{C2n} \cdot \mathit{MC2n} \cdot \mathit{vdCn}}{\mathit{C1n} + \mathit{C2n} + \mathit{kdCn}} - \mathit{C2n} \cdot \mathit{vdC1P}\\
\end{align*}
  

% \tablehead{}
\begin{tabular}{rl}
\toprule Parameter & Value\\\midrule
$vtp $  & 0.195   \\
$vtc1$  & 0.131   \\
$vtc2$  & 0.114   \\
$knp $  & 0.425   \\
$knc1$  & 0.259   \\
$vdp $  & 0.326   \\
$vdc1$  & 0.676   \\\bottomrule
\end{tabular}\hfil
%
\begin{tabular}{rl}
\toprule Parameter & Value\\\midrule
$vdc2$ & 0.608   \\
$kdp $ & 0.0115  \\
$kdc1$ & 1.15    \\
$vdP $ & 2.97    \\
$vdC1$ & 0.0338  \\
$vdC2$ & 1.52    \\
$vdCn$ & 1.69    \\\bottomrule
\end{tabular}\hfil
%
\begin{tabular}{rl}
\toprule Parameter & Value\\\midrule
$MC2n $ & 2.02    \\
$kdP  $ & 0.101   \\
$kdC1 $ & 3.32    \\
$kdCn $ & 0.0526  \\
$vaC1P$ & 0.0406  \\
$vdC1P$ & 0.00175 \\
$ktxnp$ & 3.0     \\\bottomrule
\end{tabular}
\end{model}

\begin{model}
  \caption{The Oregonator model \cite{Field1974}, with parameters as presented
  in \cite{Gillespie1977}. Used in Fig.~S3 as an example of a limit cycle
oscillator using only mass-action kinetics.}

  The individual chemical reactions (used for stochastic simulation) are:
\begin{align*}
  X_1 + Y_2 &\xrightarrow{c_1} Y_1\\
  Y_1 + Y_2 &\xrightarrow{c_2} Z_1\\
  X_2 + Y_1 &\xrightarrow{c_3} 2Y_1 + Y_3\\
  2Y_1 &\xrightarrow{c_4} Z_2\\
  X_3 + Y_3 &\xrightarrow{c_5} Y_2
\end{align*}

These reactions can be converted to a standard ODE model, in which only the
intermediate variables $Y$ are considered, as:
\begin{align*}
  \frac{d\,\mathit{Y1}}{dt} &= \mathit{c1x1}\cdot\mathit{Y2} -
  \mathit{c2}\cdot\mathit{Y1}\cdot\mathit{Y2} + \mathit{c3x2}\cdot\mathit{Y1} -
  \mathit{c4}\cdot\mathit{Y1}^2\\
  \frac{d\,\mathit{Y2}}{dt} &= -\mathit{c1x1}\cdot\mathit{Y2} -
  \mathit{c2}\cdot\mathit{Y1}\cdot\mathit{Y2} + \mathit{c5x3}\cdot\mathit{Y3}\\
  \frac{d\,\mathit{Y3}}{dt} &= \mathit{c3x2}\cdot\mathit{Y1} -
  \mathit{c5x3}\cdot\mathit{Y3}
\end{align*}
  
\begin{center}
  \tablehead{\toprule Parameter & Value\\\midrule}
  \begin{supertabular}{rl}
    $\mathit{c1x1}$ & 2 \\
    $\mathit{c2}$ & 0.1 \\
    $\mathit{c3x2}$ & 104 \\
    $\mathit{c4}$ & 0.016 \\
    $\mathit{c5x3}$ & 26 \\
    \bottomrule\end{supertabular}
\end{center}
\end{model}


\renewcommand{\thefigure}{S\arabic{figure}}
\setcounter{figure}{0}

\begin{figure}[h!]
  \begin{center}
    \includegraphics[width=.75\textwidth]{figures/figure_S1.pdf}
    \caption{{\bfseries Convergence of finite-difference and differential
    methods.} ({\bfseries A}) Perturbations of decreasing strength to the state
    of the oscillator result in phase and amplitude response curves that match
    the differential limit. ({\bfseries B}) Temporary perturbations of
    decreasing strength to a kinetic parameter. The duration of each
    perturbation is fixed to 0.2 radians. Finite difference approximations
    closely match the differential method, which is offset in phase by 0.1
    radians to account for the nonzero pulse duration.}
\end{center}
\end{figure}

\begin{figure}[h!]
  \begin{center}
    \includegraphics[width=.75\textwidth]{figures/figure_S2.pdf}
    \caption{{\bfseries Approximation of an explicit stochastic population by
    continuous methods.} ({\bfseries A}) A population of 225 stochastic
    oscillators was simulated using the Gillespie stochastic simulation
    algorithm (SSA), see Movie S1 and S2. The mean protein
    expression, $Y$, is plotted as a function of time. A $50\%$ reduction in
    the protein translation rate to each of the oscillators is applied from
    $\hat{t} = -\nicefrac{\pi}{4} \to 0$, resulting in both single-cell and
    population-level amplitude change (red solid line). This population is
    approximated by the continuous methods described in this manuscript, in
    which the decay parameter $d=0.025$ is estimated to match the
    stochastic-induced desynchrony of the control population (black solid
    line). The initial standard deviation $\sigma_0 = 0.48$ is similarly
    matched to the stochastic population. The resulting predicted unperturbed
    and perturbed trajectories, $\bar{x}(\hat{t})$ and $\hat{x}(\hat{t})$
    respectively, closely match the stochastically modeled values.  ({\bfseries
    B}) Phase histograms for the stochastic population are shown at several
    phases, both before and after the desynchronizing perturbation.  Phase
    probability-density functions for the continuous approximation are also
    shown, with close agreement between stochastic and continuous simulations.
    This close approximation validates the use of ODE models and
  phase-diffusion populations in deriving amplitude and phase-response behavior
for networks of uncoupled cells.}
\end{center}
\end{figure}

\begin{figure}[h!]
  \begin{center}
    \includegraphics[width=\textwidth]{figures/figure_S3.pdf}
    \caption{
{\bfseries Validation of continuous methods with alternate models.}
({\bfseries A}) The Oregonator model (Model 3) was tested as an example of a mass-action limit cycle oscillator.
A population of 2000 oscillators was perturbed at size different mean phases, $\mu$, by increasing the $\mathit{c2}$ parameter by $50\%$ for $d=\pi$ (highlighted region).
Only the average $\mathit{Y2}$ variable is plotted.
({\bfseries B}) The model described in \cite{Hirota2012} (Model 2) was similarly tested at a variety of mean phases by reducing the $\mathit{vdp}$ parameter by $28.5\%$ and plotting the resulting $\mathit{c2}$ state variable (as in Fig.~5).
Parameters for the continuous approximations were found by estimating the phase, initial standard deviation, period, and phase diffusivity of the control population.
Good agreement is seen between the continuous and stochastic simulations, indicating the proposed method is suitable for predicting the population-level responses for a variety of model types.
The slightly reduced accuracy seen in the larger Model 2 demonstrates a limitation of the method, which should be considered before applying the method to extreme cases.
As Eq.~44 tabulates the response of the system from the limit cycle, systems which are perturbed from states far from the deterministic limit cycle will not be captured appropriately.
Therefore, the difference in accuracy between Model 2 and Model 3 likely comes from increased deviation about the deterministic limit cycle in a greater number of spatial dimensions.
}
\end{center}
\end{figure}

\begin{figure}[h!]
  \begin{center}
    \includegraphics[width=.75\textwidth]{figures/figure_S4.pdf}
    \caption{
{\bfseries Evidence of single-cell and population-level amplitude change.}
({\bfseries A}) The simulated control trajectory as shown in Fig.~5 is plotted again as $\bar{x}(\tilde{t})$.
The simulated light-sensitive trajector is also replotted as $\hat{x}(\tilde{t})$, which incorperates both single-cell and population-level effects.
The difference between this trajectory and the steady-state perturbed trajectory following both pulses, $\hat{x}_{ss}(\tilde{t})$, demonstrates the effect of single-cell level amplitude change in this particular case. 
Single-cell amplitudes are greatly increased following the first pulse, and altered significantly following the second.
({\bfseries B})
Comparison of the population densities for the control (dashed) and perturbed (solid) simulations.
The perturbed population is greatly desynchronized following the first light pulse, as indicated by its flatness relative to the control population. 
Following the second pulse, the perturbed population is more concentrated than the control, indicating higher amplitudes. 
These comparisons demonstrate the changes in population synchrony which occur due to the light pulse. 
}
\end{center}
\end{figure}

\clearpage
{\bfseries Movie S1:} Stochastic simulation of a two-state limit cycle oscillator.
A population of 225 stochastic oscillators was simulated using the Gillespie stochastic simulation algorithm (SSA).
The deterministic limit cycle, $x^\gamma(\hat{t})$, is shown in black (left), and the current location of individual stochastic oscillators are shown by green dots.
The population average, plotted in both the state (left) and time (right) domains, is shown in red.
Stochastic noise drives the population to gradually desynchronize, resulting in damped oscillations of the average protein level.
\\[2ex]

{\bfseries Movie S2:} Perturbation to a population of stochastic oscillators.
The population of oscillators is perturbed by a $50\%$ reduction in the protein translation rate to each of the oscillators from $\hat{t} = -\nicefrac{\pi}{4} \to 0$ (yellow frames), resulting in both single-cell and population-level amplitude change.
The oscillators quickly recover to the limit cycle (single-cell level amplitude change), but the population is left desynchronized (population-level amplitude change).

\renewcommand{\refname}{Supporting References}
\bibliographystyle{biophysj.bst}
\bibliography{condensed_library.bib}
\end{document}
